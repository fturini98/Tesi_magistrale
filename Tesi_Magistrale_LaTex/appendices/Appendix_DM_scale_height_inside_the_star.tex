\section{DM scale height inside the star}
\label{App:DM_Scale_height}
Following the notation in reference \cite{SpergelPress_Cond}, and assuming that the captured DM is in thermal equilibrium with the star, it populates a small core region where it's roughly valid the equation:
\begin{equation*}
\frac{3}{2}k_B T_c=m_{\chi}V(r_{\chi})
\end{equation*}
where $T_c$ is the central temperature of the star. Now, making a second approximation, we can consider this star region as an equal density region at $\rho_c$, obtaining that:
\begin{equation*}
    V(r)\simeq \frac{2\pi}{3}G\rho_cr^2
\end{equation*}
so these two equations give:
\begin{equation*}
    r_{\chi}=\Big(\frac{9}{4\pi}\frac{k_BT_c}{G\rho_c m_{\chi}}\Big)^{\frac{1}{2}}
\end{equation*}

This scale height is useful to have a first comprehension of the DM distribution inside the star. It's possible to define the DM mean free path as:
\begin{equation*}
    l_{\chi}=\frac{1}{\sum n_{i}\sigma_{i}}
\end{equation*}
Where it was considered the presence of different nuclei inside the star. Therefore we could compare the dimension of these two variables and define two apposite regimes, one, where the mean free path is much smaller than the scale height of DM core and the other one where the DM core scale height is much smaller than the mean free path. In the first one, which is called also the local thermal equilibrium regime, the DM makes many interactions inside the core reaching the equilibrium whit the stellar structure locally,  in the second, the DM is practically free, so it travels along all the stellar structure without reaching that.